\documentclass[a4paper]{article}
\usepackage[english]{babel}
\usepackage[utf8x]{inputenc}
\usepackage{booktabs}
\usepackage{tabu}
\usepackage[T1]{fontenc}
\usepackage[a4paper,top=3cm,bottom=2cm,left=3cm,right=3cm,marginparwidth=1.75cm]{geometry}
\usepackage{amsmath}
\usepackage{graphicx}
\usepackage[colorinlistoftodos]{todonotes}
\usepackage[colorlinks=true, allcolors=blue]{hyperref}

\title{Material designing and analysing based on  density functional theory calculation and machine learning}
\author{Jie Huang}
%\date{\today}

\begin{document}
\maketitle

\section*{Summary}
% 为了进一步锻炼科研能力,发挥我的专业特长,我选择加入澳门大学应用物理及材料工程系蔡永青教授的课题组,从事第一性原理计算的研究方向。我的研究兴趣是将机器学习应用到交叉学科。我在研究生阶段从事的是机器学习和凝聚态物理结合的方向。在研究生阶段积累了充分的机器学习的应用经验,并且在研究过程中使用到第一性原理计算。因此在博士期间我计划使用第一性原理计算来模拟材料,结合机器学习工具来设计材料和分析材料的性质。材料信息学作为一门新兴的学科,机器学习在其中有着举足轻重的地位。蔡永青教授是计算材料学与计算凝聚态物理的专家,他的其中一个研究兴趣是正是材料信息学和人工智能。因此我想要加入蔡永青教授的课题组从事科研。

To further exercise my research ability and give full play to my professional expertise, I'd like to join the research group of Prof. Yongqing Cai in the Department of Applied Physics and Materials Engineering of the University of Macau.

My research interest is material designing and analyzing based on  density functional theory (DFT) calculation and machine learning. I was engaged in the intersection between machine learning and condensed matter physics as a  graduate student. I  have accumulated sufficient experience in applications of machine learning, and DFT calculations in the research process. 

Prof. Cai  is an expert in computational materials science and computational condensed matter physics. One of his research interests is material informatics where machine learning plays an important role. Therefore, I want to join Prof. Cai's research group to engage in scientific research. During my Ph.D. period, I plan to use DFT calculations to simulate materials, combined with machine learning to design materials and analyze the properties of materials. 

\section*{Background}
The rapid development of big data and machine learning has resulted in new data-driven materials research, which has achieved substantial progress \cite{xie2021}. The amount of data being generated by experiments and simulations has given rise to the fourth paradigm of science: data-driven science \cite{hey2009the},  and it unifies the first three paradigms of theory, experiment, and computation as shown in Figure \ref{fig:datadriven}.  This massive amount of data needs to be stored and interpreted in order to advance the materials science field.  Identifying correlations and patterns from large amounts of complex data is being performed by machine learning algorithms for decades \cite{Schleder2019}.  It is increasingly becoming popular in the field of materials science as well and has led to the emergence of the new field of materials informatics \cite{Rajan2005, Agrawal2016}, which aims to discover the relations between known standard features and materials properties. 

\begin{figure}
	\centering
	\includegraphics[width=0.7\linewidth]{imgs/Data_driven}
	\caption[]{The four science paradigms: empirical, theoretical, computational, and data-driven\cite{Schleder2019, Agrawal2016}.}
	\label{fig:datadriven}
\end{figure}

%Machine learning is a class of methods for automated data analysis, which are capable of detecting patterns in data. These extracted patterns can be used to predict unknown data or to assist in decision-making processes under uncertainty. The traditional definition states that the machine learning, i.e. progressive performance improvement on a task directed by available data, takes place without being explicitly programmed. This research field evolved from the broader area of artificial intelligence, inspired by the 1950s developments in statistics, computer science and technology, and neuroscience. \cite{Schleder2019} ML models can be supervised, semi-supervised or unsupervised, depending on the type of available training data. In supervised learning, the training data consist of sets of input and associated output values. In other words, labelled training samples are required. On the other hand, if the training dataset contains unlabeled samples, unsupervised learning can be used in ordered to identify trends and patterns in the data. Semisupervised learning can be used for large datasets with partially missing labels. \cite{Pilania2021}

%Inspired by the success of applied information sciences such as bioinformatics, the application of machine learning and data-driven techniques to materials science developed into a new sub-field called ‘Materials Informatics’ \cite{Rajan2005}, which aims to discover the relations between known standard features and materials properties. These features are usually restricted to the structure, composition, symmetry, and properties of the constituent elements.

% 我的研究方向可能与下面几个例子相关。
The following are two topics related to my future research.

1. High-entropy alloys. High-entropy alloys (HEAs) are alloys with five or more principal elements. Due to the distinct design concept, these alloys often exhibit unusual properties. Thus, there has been significant interest in these materials, leading to an emerging yet exciting new field \cite{Tsai2014}.  Identifying single-phase HEAs is extremely important to understanding HEA formation and their intrinsic properties, but the lack of effective guidelines has hindered their discovery. \cite{Gao2016} Besides, the experimental process is still limited by high costs and time-consuming synthesis procedures. \cite{Li2020} The work of Chang et al. \cite{Chang2019} utilizes an artificial neural network (ANN) to predict the composition of HEAs  in order to achieve the highest hardness in the system. A simulated annealing algorithm is integrated with the ANN to optimize the composition. This work  demonstrates that, by applying the machine learning method, new compositions of  HEAs can be obtained, exhibiting hardness values higher than the best literature value for the same alloy system. 


2. Nanostructures for Phonon Transport: In another work, Ju et al. \cite{Ju2017} identified the Si/Ge composite interfacial structures that minimize or maximize the interfacial thermal conductance across Si-Si and Si-Ge interfaces by the developed framework combining the atomistic Green function and Bayesian optimization methods. The optimal structures are obtained by calculating only a few percent of the total candidate structures, considerably saving computational resources. The validity and capability of the method are demonstrated by identifying the thin interfacial structures with the optimal Si/Ge configurations among all the possible candidates.
% 设计高热导率的材料是解决电子器件散热的关键。电子器件需要散热性好的材料,这就要求我们设计出先进的热管理材料和器件,从而解决我国电子器件散热的瓶颈。这也是符合我国电子器件战略发展要求的。
Designing materials with high thermal conductivity is the key to solving the problem of heat dissipation of electronic devices. Electronic devices need materials with good heat dissipation, which requires us to design advanced thermal management materials and devices to solve the bottleneck of heat dissipation of electronic devices in our country. This is also in line with the national strategic development requirements of electronic devices.

% 在新材料的设计问题上,由于探索空间巨大。因此找到找到一种方法加速材料发现过程是紧迫的。上述提到的这些应用说明了机器学习用在材料科学是可行的。未来随着材料数据量的增加和计算机算力的不断提高。机器学习在材料科学中将会有着越来越重要的作用。但是,不同于机器视觉等计算机领域,材料科学中的机器学习应用有其特殊指出。 首先一点重要的数据量的区别,在材料科学领域的数据集的大小()是远远小于传统机器学习领域的 (>$10^6$)。原因是材料科学领域的数据通常需要通过实验得到,这些数据的获取不是那么容易。另一方面,材料领域的机器学习有很大一部分是寻找数据集中不存在的例子。而不仅仅是数据中的统计规律。因此材料领域的机器学习有很大一部分研究的是如何使用最少的尝试次数来得到我们想要的结构或者材料。上面提到的两点即使材料信息学的机遇同样也是挑战。

With regard to the design of new materials, the space for exploration is huge. Therefore, it is urgent to find a way to speed up the material discovery process. Theoretical and calculation research activities play an increasingly-important role in materials science. The combination of computer simulation and experimental data contributes to a better understanding of the physical mechanism, which therefore enables the prediction of unknown data.  The applications mentioned above show that machine learning is feasible for materials science. In the future, with the increase of the amount in material data and the continuous improvement of computer computing power, machine learning will play an increasingly important role in materials science. However, unlike traditional computer fields such as computer vision, machine learning applications in materials science have their characteristics. First of all, there is a huge difference in the amount of data. The size of the data available in the field of materials science (< $10^3$) is much smaller than that in the field of traditional machine learning (>$10^6$). The reason is that data in the field of materials science usually needs to be obtained through experiments, and the acquisition of these data is expensive. On the other hand, a large part of machine learning in the material field is looking for examples that do not exist in the data set. Therefore, a large part of machine learning in the material field studies how to use the least number of attempts to get the structure or material we desire. 

\section*{Objectives and Methods}
%我的研究目标主要有两个:
%1. 用机器学习辅助材料的设计,用于发现新的材料。
%2. 用机器学习方法对第一性原理计算的模拟数据做分析,进而探索材料的性质。
The goal of this research is to use DFT calculations and machine learning to 
\begin{itemize}
	\item aid the design of materials, and discover new materials;
	\item analyze the simulated data and explore the properties of the materials.
\end{itemize}

% 第一性原理模拟
The methods used in this research mainly include the following two aspects.

Method 1: DFT is a mature theory that is currently the undisputed choice of method for electronic structure calculations. A number of papers and reviews are presented in the literature \cite{Jones2015, Burke2012, Perdew2010}, facilitating the widespread of the theory and, thus, the entry of researchers into the field of computational solid-state physics, materials science, and quantum chemistry. 
% 采用DFT计算模拟的原因有: 1. DFT计算能够从一点程度上解决材料学科中数据量不足的问题。2. 在研究生阶段我做过DFT模拟计算,对于模拟流程较为熟悉。
On the one hand, DFT calculation can solve the problem of insufficient data in materials science to a certain extent. On the other hand,  I am familiar with the DFT simulations.
% 因此在我未来的研究中,我将采用DFT计算的方法对材料进行模拟,例如二维材料。研究其声子特征和热传导性能等。
Therefore, in my research, I will use the DFT method to simulate materials, such as two-dimensional materials \cite{Cai2014, Cai2015, Yuan2020}, which is one topic of Prof. Cai's projects, study its phonon characteristics and thermal conductivity. 

% 添加Porf. Cai



% 机器学习方法
%在辅助材料设计方面,结合前人的经验。我可能会用到的工具有贝叶斯优化,蒙特卡洛树搜索等方法。强化学习作为一种提供最优策略的方法,在辅助材料设计方面应该会大有可为。 因此我计划将强化学习用到材料设计当中。至于具体的分析工具,如今的数据分析软件有很多,我会选择我擅长的Python语言结合机器学习软件库Tensorflow等用来做机器学习模型的建模和数据分析。

Method 2: As for machine learning, combined with previous experience, the tools I may use include ANN, Monte Carlo tree search, Bayesian optimization.  Reinforcement learning, as a method of providing optimal strategies, is promising in the material designing. So I plan to explore the methods of  reinforcement learning in material designing. Thanks to many data analysis software nowadays, I will choose the Python that I am good at combined with the widely used machine learning software library, like Tensorflow, etc. to build machine learning models and analyze data.

\bibliographystyle{unsrt}
\bibliography{refs}

\end{document}
